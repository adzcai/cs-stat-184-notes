\providecommand{\main}{..}

\documentclass[\main/main]{subfiles}

\newcommand{\Nex}{N_{\text{explore}}}
\newcommand{\muv}{\boldsymbol{\mu}}

\begin{document}

\tableofcontents
    
\chapter{Bandits}


The \textbf{multi-armed bandits} (MAB) setting is a simple but powerful setting for studying the basic challenges of RL. In this setting, an agent repeatedly chooses from a fixed set of actions, called \textbf{arms}, each of which has an associated reward distribution. The agent's goal is to maximize the total reward it receives over some time period.

\rltable{None}{Finite}{Stochastic}

In particular, we'll spend a lot of time discussing the \textbf{Exploration-Exploitation Tradeoff}: should the agent choose new actions to learn more about the environment, or should it choose actions that it already knows to be good?

\begin{example}{Online advertising}{advertising}
    Let's suppose you, the agent, are an advertising company. You have $K$ different ads that you can show to users; For concreteness, let's suppose there's just a single user. You receive $1$ reward if the user clicks the ad, and $0$ otherwise. Thus, the unknown \emph{reward distribution} associated to each ad is a Bernoulli distribution defined by the probability that the user clicks on the ad. Your goal is to maximize the total number of clicks by the user.
\end{example}

\begin{example}{Clinical trials}{clinical_trials}
    Suppose you're a pharmaceutical company, and you're testing a new drug. You have $K$ different dosages of the drug that you can administer to patients. You receive $1$ reward if the patient recovers, and $0$ otherwise. Thus, the unknown \emph{reward distribution} associated to each dosage is a Bernoulli distribution defined by the probability that the patient recovers. Your goal is to maximize the total number of patients that recover.
\end{example}

In this chapter, we will introduce the multi-armed bandits setting, and discuss some of the challenges that arise when trying to solve problems in this setting. We will also introduce some of the key concepts that we will use throughout the book, such as regret and exploration-exploitation tradeoffs.

\section{Introduction}

The name ``multi-armed bandits'' comes from slot machines in casinos, which are often called ``one-armed bandits'' since they have one arm (the lever) and take money from the player.

Let $K$ denote the number of arms. We'll label them $0, \dots, K-1$ and use \emph{superscripts} to indicate the arm index; since we seldom need to raise a number to a power, this hopefully won't cause much confusion. For simplicity, we'll assume rewards are \emph{bounded} between $0$ and $1$. Then each arm has an unknown reward distribution $\nu^k \in \Delta([0, 1])$ with mean $\mu^k = \E_{r \sim \nu^k} [r]$.

In pseudocode, the agent's interaction with the MAB environment can be described by the following process:

\begin{algorithmic}
% \Comment multi-armed bandits
\For{$t = 0, \dots, T$}
    \State Agent chooses $a_t \in [K]$
    \State Agent receives $r_t \sim \nu^{a_t}$
    \State Agent updates its internal state
\EndFor
\end{algorithmic}

What's the \emph{optimal} strategy for the agent, i.e. the one that achieves the highest expected reward? Convince yourself that the agent should try to always pull the arm with the highest expected reward $\mu^\star := \max_{k \in [K]} \mu^k$.

The goal, then, can be rephrased as to minimize the \textbf{regret}, defined below:

\begin{definition}{Regret}{regret}
    
The agent's \textbf{regret} after $T$ timesteps is defined as
\begin{equation}
    \text{Regret}_T := \sum_{t=0}^{T-1} \mu^\star - \mu^{a_t}
\end{equation}
Note that this depends on the \emph{true means} of the pulled arms, \emph{not} the actual observed rewards. We typically think of this as a random variable where the randomness comes from the agent's strategy (i.e. the sequence of actions $a_0, \dots, a_{T-1}$).

Throughout the chapter, we will try to upper bound the regret of various algorithms in two different senses:
\begin{enumerate}
    \item Upper bound the \emph{expected} regret, i.e. show $\E[\text{Regret}_T] \le M_T$.
    \item Find a high-probability upper bound on the regret, i.e. show $\P(\text{Regret}_T \le M_{T, \delta}) \ge 1-\delta$.
\end{enumerate}
Note that these two different approaches say very different things about the regret. The first approach says that the \emph{average} regret is at most $M_T$. However, the agent might still achieve higher regret on many runs. The second approach says that, \emph{with high probability}, the agent will achieve regret at most $M_{T, \delta}$. However, it doesn't say anything about the regret in the remaining $\delta$ fraction of runs, which might be arbitrarily high.

\end{definition}

We'd like to achieve \textbf{sublinear regret} in expectation, i.e. $\E[\text{Regret}_T] = o(T)$. That is, as we learn more about the environment, we'd like to be able to exploit that knowledge to achieve higher rewards.

The rest of the chapter comprises a series of increasingly sophisticated MAB algorithms.

\section{Pure exploration (random guessing)}

A trivial strategy is to always choose arms at random (i.e. ``pure exploration'').
\begin{definition}{Pure exploration}{pure_exploration}
\begin{algorithmic}
\For{$t \gets 0$ to $T-1$}
    \State Choose $a_t \sim \text{Unif}([K])$
    \State Observe $r_t \sim \nu^{a_t}$
\EndFor
\end{algorithmic}
\end{definition}
Note that \[
    \E_{a_t \sim \text{Unif}([K])}[\mu^{a_t}] = \bar \mu = \frac{1}{K} \sum_{k=1}^K \mu^k
\] so the expected regret is simply
\begin{align*}
    \E[\text{Regret}_T] &= \sum_{t=0}^{T-1} \E[\mu^\star - \mu^{a_t}] \\
    &= T (\mu^\star - \bar \mu) > 0.
\end{align*}
This scales as $\Theta(T)$, i.e. \emph{linear} in the number of timesteps $T$. There's no learning here: the agent doesn't use any information about the environment to improve its strategy.

\section{Pure greedy}

How might we improve on pure exploration? Instead, we could try each arm once, and then commit to the one with the highest observed reward. We'll call this the \textbf{pure greedy} strategy.

\begin{definition}{Pure greedy}{pure_greedy}
\begin{algorithmic}
    \For{$k \gets 0$ to $K-1$}
        \Comment Exploration phase
        \State Observe $r^k \sim \nu^k$
    \EndFor
    \State $\hat k \gets \argmax_{k \in [K]} r^k$
    \For{$t \gets K$ to $T-1$}
        \Comment Exploitation phase
        \State Observe $r_t \sim \nu^{\hat k}$
    \EndFor
\end{algorithmic}
Note we've used superscripts $r^k$ during the exploration phase to indicate that we observe exactly one reward for each arm. Then we use subscripts $r_t$ during the exploitation phase to indicate that we observe a sequence of rewards from the chosen greedy arm $\hat k$.
\end{definition}

How does the expected regret of this strategy compare to that of pure exploration? We'll do a more general analysis in the following section. Now, for intuition, suppose there's just $K=2$ arms, with Bernoulli reward distributions with means $\mu^0 > \mu^1$.

Let's let $r^0$ be the random reward from the first arm and $r^1$ be the random reward from the second. If $r^0 > r^1$, then we achieve zero regret. Otherwise, we achieve regret $T(\mu^0 - \mu^1)$. Thus, the expected regret is simply:

\begin{align*}
    \E[\text{Regret}_T] &= \P(r^0 < r^1) \cdot T(\mu^0 - \mu^1) + c \\
    &= (1 - \mu^0) \mu^1 \cdot T(\mu^0 - \mu^1) + c
\end{align*}

Which is still $\Theta(T)$, the same as pure exploration! Can we do better?

\section{Explore-then-commit}

We can improve the pure greedy algorithm as follows:
let's reduce the variance of the reward estimates by pulling each arm $\Nex > 1$ times before committing. This is called the \textbf{explore-then-commit} strategy.
\begin{definition}{Explore-then-commit}{etc}
\begin{algorithmic}
\State \textbf{Input:} $\Nex \le T/K$
\For{$k \gets 0$ to $K-1$}
    \Comment Exploration phase
    \For{$i \gets 0$ to $\Nex-1$}
        \State $r^k_i \sim \nu^k$
    \EndFor
    \State $\hat \mu^k \gets \frac{1}{\Nex} \sum_{i=0}^{\Nex-1} r^k_i$
\EndFor
\State $\hat k \gets \argmax_k (\hat \mu^k)$
\For{$t \gets \Nex K$ to $T-1$}
    \Comment Exploitation phase
    \State $r_t \sim \nu^{\hat k}$
\EndFor
\end{algorithmic}
\end{definition}
(Note that the ``pure greedy'' strategy is just the special case where $\Nex = 1$.)

\subsection{ETC regret analysis} \label{sec:etc-regret-analysis}

Let's analyze the expected regret of this strategy by splitting it up into the exploration and exploitation phases.

\paragraph*{Exploration phase.} This phase takes $\Nex K$ timesteps. Since at each step we incur at most $1$ regret, the total regret is at most $\Nex K$.

\paragraph*{Exploitation phase.} This will take a bit more effort. We'll prove that for any total time $T$, we can choose $\Nex$ such that with arbitrarily high probability, the regret is sublinear. We know the regret from the exploitation phase is
\[
    T_{\text{exploit}} (\mu^\star - \mu^{\hat k}) \qquad \text{where} \qquad T_{\text{exploit}} := T - \Nex K.
\]
So we'd like to bound $\mu^\star - \mu^{\hat k} = o(1)$ (as a function of $T$) in order to achieve sublinear regret. How can we do this?

Let's define $\Delta^k = \hat \mu^k - \mu^k$ to denote how far the mean estimate for arm $k$ is from the true mean. How can we bound this quantity? We'll use the following useful inequality for i.i.d. bounded random variables:

\begin{theorem}{Hoeffding's inequality}{hoeffding}
    Let $X_0, \dots, X_{n-1}$ be i.i.d. random variables with $X_i \in [0, 1]$ almost surely for each $i \in [n]$. Then for any $\delta > 0$,
    
    \begin{equation}
        \P\left( \left| \frac{1}{n} \sum_{i=1}^n (X_i - \E[X_i]) \right| > \sqrt{\frac{\ln(2/\delta)}{2n}} \right) \le \delta.
    \end{equation}
\end{theorem}

(The proof of this inequality is beyond the scope of this book.) We can apply this directly to the rewards for a given arm $k$, since the rewards from that arm are i.i.d.:
\begin{equation}
    \P\left(|\Delta^k | > \sqrt{\frac{\ln(2/\delta)}{2\Nex}} \right) \le \delta. \label{eq:hoeffding-etc}
\end{equation}
But note that we can't apply this to arm $\hat k$ directly since $\hat k$ is itself a random variable. Instead, we need to ``uniform-ize'' this bound across \emph{all} the arms, i.e. bound the error across all the arms simultaneously, so that the resulting bound will apply \emph{no matter what} $\hat k$ ``crystallizes'' to.

The \textbf{union bound} provides a simple way to do this:

\begin{theorem}{Union bound}{union_bound}
    Consider a set of events $A_0, \dots, A_{n-1}$. Then \[
        \P(\exists i \in [n]. A_i) \le \sum_{i=0}^{n-1} \P(A_i).
    \]
    In particular, if $\P(A_i) \ge 1 - \delta$ for each $i \in [n]$, we have \[
        \P(\forall i \in [n]. A_i) \ge 1 - n \delta.
    \]
\end{theorem}

\textbf{Exercise:} Prove the second statement above.

Applying the union bound across the arms for the l.h.s. event of \ref*{eq:hoeffding-etc}, we have
\begin{align*}
    \P\left( \forall k \in [K], |\Delta^k | \le \sqrt{\frac{\ln(2/\delta)}{2\Nex}} \right) &\ge 1-K\delta
\end{align*}
Then to apply this bound to $\hat k$ in particular, we can apply the useful trick of ``adding zero'':
\begin{align*}
    \mu^{k^\star} - \mu^{\hat k} &= \mu^{k^\star} - \mu^{\hat k} + (\hat \mu^{k^\star} - \hat \mu^{k^\star}) + (\hat \mu^{\hat k} - \hat \mu^{\hat k}) \\
    &= \Delta^{\hat k} - \Delta^{k^*} + \underbrace{(\hat \mu^{k^\star} - \hat \mu^{\hat k})}_{\le 0 \text{ by definition of } \hat k} \\
    &\le 2 \sqrt{\frac{\ln(2K/\delta')}{2\Nex}} \text{ with probability at least } 1-\delta'
\end{align*}
where we've set $\delta' = K\delta$.
Putting this all together, we've shown that, with probability $1 - \delta'$,
\[
    \text{Regret}_T \le \Nex K + T_{\text{exploit}} \cdot \sqrt{\frac{2\ln(2K/\delta')}{\Nex}}.
\]

Note that it suffices for $\Nex$ to be on the order of $\sqrt{T}$ to achieve sublinear regret. In particular, we can find the optimal $\Nex$ by setting the derivative of the r.h.s. to zero:
\begin{align*}
    0 &= K - T_{\text{exploit}} \cdot \frac{1}{2} \sqrt{\frac{2\ln(2K/\delta')}{\Nex^3}} \\
    \Nex &= \left( T_{\text{exploit}} \cdot \frac{\sqrt{\ln(2K/\delta')/2}}{K} \right)^{2/3}
\end{align*}
Plugging this into the expression for the regret, we have (still with probability $1-\delta'$)
\begin{align*}
    \text{Regret}_T &\le 3 T^{2/3} \sqrt[3]{K \ln(2K/\delta') / 2} \\
    &= \tilde{O}(T^{2/3} K^{1/3}).
\end{align*}

The ETC algorithm is rather ``abrupt'' in that it switches from exploration to exploitation after a fixed number of timesteps. In practice, it's often better to use a more gradual transition, which brings us to the \emph{epsilon-greedy} algorithm.

\section{Epsilon-greedy}

Instead of doing all of the exploration and then all of the exploitation separately -- which additionally requires knowing the time horizon beforehand -- we can instead interleave exploration and exploitation by, at each timestep, choosing a random action with some probability. We call this the \textbf{epsilon-greedy} algorithm.

\begin{definition}{Epsilon-greedy}{epsilon_greedy}
\begin{algorithmic}
\State \textbf{Input:} $\epsilon : \mathbb{N} \to [0, 1]$
\State $S^k \gets 0$ for each $k \in [K]$ \Comment{Total reward for arm $k$}
\State $N^k \gets 0$ for each $k \in [K]$ \Comment{Number of pulls for arm $k$}
\For{$t \gets 1$ to $T$}
    \If{$\text{random}() < \epsilon(t)$}
        \State $k \sim \text{Unif}([K])$
        \Comment{Exploration}
    \Else
        \State $k \gets \argmax_k \left(\frac{S^k}{N^k}\right)$
        \Comment{Exploitation}
    \EndIf
    \State $r_t \sim \nu^k$
    \State $S^k \gets S^k + r_t$
    \State $N^k \gets N^k + 1$
\EndFor
\end{algorithmic}
\end{definition}

Note that we let $\epsilon$ vary over time. In particular we might want to gradually \emph{decrease} $\epsilon$ as we learn more about the reward distributions over time.

It turns out that setting $\epsilon_t = \sqrt[3]{K \ln(t)/t}$ also achieves a regret of $\tilde O(t^{2/3} K^{1/3})$ (ignoring the logarithmic factors). (We will not prove this here.)

% TODO proof
% \todo{Insert optimal epsilon and regret analysis / high probability bound}

In ETC, we had to set $\Nex$ based on the total number of timesteps $T$. But the epsilon-greedy algorithm actually handles the exploration \emph{automatically}: the regret rate holds for \emph{any} $t$, and doesn't depend on the final horizon $T$.

But the way these algorithms explore is rather naive: we've been exploring \emph{uniformly} across all the arms. But what if we could be smarter about it, and explore \emph{more} for arms that we're less certain about?


\section{Upper Confidence Bound (UCB)}

To quantify how \emph{certain} we are about the mean of each arm, we'll compute \emph{confidence intervals} for our estimators, and then choose the arm with the highest \emph{upper confidence bound}. This operates on the principle of \textbf{the benefit of the doubt (i.e. optimism in the face of uncertainty)}: we'll choose the arm that we're most optimistic about.

In particular, for each arm $k$ at time $t$, we'd like to compute some upper confidence bound $M^k_t$ such that $\hat \mu^k_t \le M^k_t$ with high probability, and then choose $a_t := \argmax_{k \in [K]} M^k_t$. But how should we compute $M^k_t$?

In \autoref{sec:etc-regret-analysis}, we were able to compute this bound using Hoeffding's inequality, which assumes that the number of samples is \emph{fixed}. This was the case in ETC (where we pull each arm $\Nex$ times), but in UCB, the number of times we pull each arm depends on the agent's actions, which in turn depend on the random rewards and are therefore stochastic. So we \emph{can't} use Hoeffding's inequality directly.

Instead, we'll apply the same trick we used in the ETC analysis: we'll use the \textbf{union bound} to compute a \emph{looser} bound that holds \emph{uniformly} across all timesteps and arms. Let's introduce some notation to discuss this.

Let $N^k_t$ denote the (random) number of times arm $k$ has been pulled within the first $t$ timesteps, and $\hat \mu^k_t$ denote the sample average of those pulls. That is,
\begin{align*}
    N^k_t &:= \sum_{\tau=0}^{t-1} \mathbf{1} \{ a_\tau = k \} \\
    \hat \mu^k_t &:= \frac{1}{N^k_t} \sum_{\tau=0}^{t-1} \mathbf{1} \{ a_\tau = k \} r_\tau.
\end{align*}
To achieve the ``fixed sample size'' assumption, we'll need to shift our index from \emph{time} to \emph{number of samples from each arm}. In particular, we'll define $\tilde r^k_n$ to be the $n$th sample from arm $k$, and $\tilde \mu^k_n$ to be the sample average of the first $n$ samples from arm $k$. Then, for a fixed $n$, this satisfies the ``fixed sample size'' assumption, and we can apply Hoeffding's inequality to get a bound on $\tilde \mu^k_n$.

So how can we extend our bound on $\tilde\mu^k_n$ to $\hat \mu^k_t$? Well, we know $N^k_t \le t$ (where equality would be the case if and only if we had pulled arm $k$ every time). So we can apply the same trick as last time, where we uniform-ize across all possible values of $N^k_t$:
\begin{align*}
    \P\left( \forall n \le t, |\tilde \mu^k_n - \mu^k | \le \sqrt{\frac{\ln(2/\delta)}{2n}} \right) &\ge 1-t\delta.
\end{align*}
In particular, since $N^k_t \le t$, and $\tilde \mu^k_{N^k_t} = \hat \mu^k_t$ by definition, we have
\begin{align*}
    \P\left( |\hat \mu^k_t - \mu^k | \le \sqrt{\frac{\ln(2t/\delta')}{2N^k_t}} \right) &\ge 1-\delta' \text{ where } \delta' := t \delta.
\end{align*}
This bound would then suffice for applying the UCB algorithm! That is, the upper confidence bound for arm $k$ would be \[ M^k_t := \hat \mu^k_t + \sqrt{\frac{\ln(2t/\delta')}{2N^k_t}}, \] where we can choose $\delta'$ depending on how tight we want the interval to be. A smaller $\delta'$ would give us a larger yet higher-confidence interval, and vice versa. We can now use this to define the UCB algorithm.

\begin{definition}{Upper Confidence Bound (UCB)}{ucb}
\begin{algorithmic}
\State \textbf{Input:} $\delta' \in (0, 1)$
\For{$t \gets 0$ to $T-1$}
    \State $k \gets \argmax_{k' \in [K]} \frac{S^{k'}}{N^{k'}} + \sqrt{\frac{\ln(2t/\delta')}{2 N^{k'}}}$
    \State $r_t \sim \nu^k$
    \State $S^k \gets S^k + r_t$
    \State $N^k \gets N^k + 1$
\EndFor
\end{algorithmic}
\end{definition}

\textbf{Exercise:} As written, this ignores the issue that we divide by $N^k = 0$ for all arms at the beginning. How should we resolve this issue?

Intuitively, UCB prioritizes arms where:

\begin{enumerate}
    \item $\hat \mu^k_t$ is large, i.e. the arm has a high sample average, and we'd choose it for \emph{exploitation}, and
    \item $\sqrt{\frac{\ln(2t/\delta')}{2N^k_t}}$ is large, i.e. we're still uncertain about the arm, and we'd choose it for \emph{exploration}.
\end{enumerate}

As desired, this explores in a smarter, \emph{adaptive} way compared to the previous algorithms. Does it achieve lower regret?

\subsection{UCB regret analysis}

First we'll bound the regret incurred at each timestep. Then we'll bound the \emph{total} regret across timesteps.

For the sake of analysis, we'll use a slightly looser bound that applies across the whole time horizon and across all arms. We'll omit the derivation since it's very similar to the above (walk through it yourself for practice).

\begin{align*}
    \P\left(\forall k \le K, t < T. |\hat \mu^k_t - \mu^k | \le B^k_t \right) &\ge 1-\delta'' \\
    \text{where} \quad B^k_t &:= \sqrt{\frac{\ln(2TK/\delta'')}{2N^k_t}}.
\end{align*}

Intuitively, $B^k_t$ denotes the \emph{width} of the CI for arm $k$ at time $t$. Then, assuming the above uniform bound holds (which occurs with probability $1-\delta''$), we can bound the regret at each timestep as follows:

\begin{align*}
    \mu^\star - \mu^{a_t} &\le \hat \mu^{k^*}_t + B_t^{k^*} - \mu^{a_t} && \text{applying UCB to arm } k^\star \\
    &\le \hat \mu^{a_t}_t + B^{a_t}_t - \mu^{a_t} && \text{since UCB chooses } a_t = \argmax_{k \in [K]} \hat \mu^k_t + B_t^{k} \\
    &\le 2 B^{a_t}_t && \text{since } \hat \mu^{a_t}_t - \mu^{a_t} \le B^{a_t}_t \text{ by definition of } B^{a_t}_t \\
\end{align*}

Summing this across timesteps gives

\begin{align*}
    \text{Regret}_T &\le \sum_{t=0}^{T-1} 2 B^{a_t}_t \\
    &= \sqrt{2\ln(2TK/\delta'')} \sum_{t=0}^{T-1} (N^{a_t}_t)^{-1/2} \\
    \sum_{t=0}^{T-1} (N^{a_t}_t)^{-1/2} &= \sum_{t=0}^{T-1} \sum_{k=1}^K \mathbf{1}\{ a_t = k \} (N^k_t)^{-1/2} \\
    &= \sum_{k=1}^K \sum_{n=1}^{N_T^k} n^{-1/2} \\
    &\le K \sum_{n=1}^T n^{-1/2} \\
    \sum_{n=1}^T n^{-1/2} &\le 1 + \int_1^T x^{-1/2} \ \mathrm{d}x \\
    &= 1 + (2 \sqrt{x})_1^T \\
    &= 2 \sqrt{T} - 1 \\
    &\le 2 \sqrt{T} \\
\end{align*}

Putting everything together gives
\begin{align*}
    \text{Regret}_T &\le 2 K \sqrt{2T \ln(2TK/\delta'')} && \text{with probability } 1-\delta'' \\
    &= \tilde O(K\sqrt{T})
\end{align*}


In fact, we can do a more sophisticated analysis to trim off a factor of $\sqrt{K}$ and show $\text{Regret}_T = \tilde O(\sqrt{TK})$.

\subsection{Lower bound on regret (intuition)}

Is it possible to do better than $\Omega(\sqrt{T})$ in general? In fact, no! We can show that any algorithm must incur $\Omega(\sqrt{T})$ regret in the worst case. We won't rigorously prove this here, but the intuition is as follows.

The Central Limit Theorem tells us that with $T$ i.i.d. samples from some distribution, we can only learn the mean of the distribution to within $\Omega(1/\sqrt{T})$ (the standard deviation). Then, since we get $T$ samples spread out across the arms, we can only learn each arm's mean to an even looser degree.

That is, if two arms have means that are within about $1/\sqrt{T}$, we won't be able to confidently tell them apart, and will sample them about equally. But then we'll incur regret \[ \Omega((T/2) \cdot (1/\sqrt{T})) = \Omega(\sqrt{T}). \]

% TODO find citation

\section{Thompson sampling and Bayesian bandits}

So far, we've treated the parameters $\mu^0, \dots, \mu^{K-1}$ of the reward distributions as \emph{fixed}. Instead, we can take a \textbf{Bayesian} approach where we treat them as random variables from some \textbf{prior distribution}. Then, upon pulling an arm and observing a reward, we can simply \emph{condition} on this observation to exactly describe the \textbf{posterior distribution} over the parameters. This fully describes the information we gain about the parameters from observing the reward.

From this Bayesian perspective, the \textbf{Thompson sampling} algorithm follows naturally: just sample from the distribution of the optimal arm, given the observations!
\begin{definition}{Thompson sampling}{thompson_sampling}
\begin{algorithmic}
\State \textbf{Input:} the prior distribution $\pi \in \Delta([0, 1]^K)$
\For{$t \gets 0$ to $T-1$}
    \State $\muv \sim \pi(\cdot \mid a_0, r_0, \dots, a_{t-1}, r_{t-1})$
    \State $a_t \gets \argmax_{k \in [K]} \mu^k$
    \State $r_t \sim \nu^{a_t}$
    \Comment{Observe reward}
\EndFor
\end{algorithmic}
\end{definition}
In other words, we sample each arm proportionally to how likely we think it is to be optimal, given the observations so far.
This strikes a good exploration-exploitation tradeoff: we explore more for arms that we're less certain about, and exploit more for arms that we're more certain about.
Thompson sampling is a simple yet powerful algorithm that achieves state-of-the-art performance in many settings.

\begin{example}{Bayesian Bernoulli bandit}{bayesian_bernoulli}
    We've often been working in the Bernoulli bandit setting, where arm $k$ yields a reward of $1$ with probability $\mu^k$ and no reward otherwise. The vector of success probabilities $\muv = (\mu^1, \dots, \mu^K)$ thus describes the entire MAB.
    
    Under the Bayesian perspective, we think of $\muv$ as a \emph{random} vector drawn from some prior distribution $\pi(\muv)$. For example, we might have $\pi$ be the Uniform distribution over the unit hypercube $[0, 1]^K$, that is, \[
        \pi(\muv) = \begin{cases}
            1 & \text{if } \muv \in [0, 1]^K \\
            0 & \text{otherwise}
        \end{cases}
    \]
    Then, upon viewing some reward, we can exactly calculate the \textbf{posterior} distribution of $\muv$ using Bayes's rule (i.e. the definition of conditional probability):
    \begin{align*}
        \P(\muv \mid a_0, r_0) &\propto \P(r_0 \mid a_0, \muv) \P(a_0 \mid \muv) \P(\muv) \\
        &\propto (\mu^{a_0})^{r_0} (1 - \mu^{a_0})^{1-r_0}.
    \end{align*}
    This is the PDF of the $\text{Beta}(1 + r_0, 1 + (1 - r_0))$ distribution, which is a conjugate prior for the Bernoulli distribution. That is, if we start with a Beta prior on $\mu^k$ (note that $\text{Unif}([0, 1]) = \text{Beta}(1, 1)$), then the posterior, after conditioning on samples from $\text{Bern}(\mu^k)$, will also be Beta. This is a very convenient property, since it means we can simply update the parameters of the Beta distribution upon observing a reward, rather than having to recompute the entire posterior distribution from scratch.
\end{example}


It turns out that asymptotically, Thompson sampling is optimal in the following sense. Lai and Robbins \cite{lai_asymptotically_1985} prove an \emph{instance-dependent} lower bound that says for \emph{any} bandit algorithm,
\[
    \liminf_{T \to \infty} \frac{\E[N_T^k]}{\ln(T)} \ge \frac{1}{\text{KL}(\mu^k \parallel \mu^\star)}
\]
where \[
    \text{KL}(\mu^k \parallel \mu^\star) = \mu^k \ln \frac{\mu^k}{\mu^\star} + (1 - \mu^k) \ln \frac{1 - \mu^k}{1 - \mu^\star}
\] measures the \textbf{Kullback-Leibler divergence} from the Bernoulli distribution with mean $\mu^k$ to the Bernoulli distribution with mean $\mu^\star$.
It turns out that Thompson sampling achieves this lower bound with equality! That is, not only is the error \emph{rate} optimal, but the \emph{constant factor} is optimal as well.

\iffalse
\section{Gittins index}

\fi

\section{Contextual bandits}

In the above MAB environment, the reward distribution of the arms remains constant for each action that we take.
However, in many real-world settings, we might receive additional information that affects the reward distributions of the arms.
For example, in the online advertising case where each arm corresponds to an ad we could show the user,
we might receive information about the user's preferences that changes how likely they are to click on a given ad.
We can model such environments using \textbf{contextual bandits}.
At each timestep $t$, a new \emph{context} $x_t$ is drawn from some distribution $\nu_{\text{x}}$.
The learner gets to observe the context, and choose an action $a_t$ according to some policy $\pi_t(x_t)$.
Then, the learner observes the reward $\nu^{a_t}(x_t)$.

Assuming our context is \emph{discrete}, we can just perform the same algorithms, treating each context-arm pair as its own arm. This gives us an enlarged MAB of $K |\mathcal{X}|$ arms.

\begin{exercise}
    Write down the UCB algorithm for this enlarged MAB. That is, write an expression for $\pi_t(x_t) = \argmax_a \dots$.
\end{exercise}

But in a situation where we have large, or even infinitely many contexts (e.g. in the case where our context is a continuous value), this becomes intractable.
How can we improve this algorithm?
Note that this treats the different contexts as entirely unrelated to each other.
But often the contexts are \emph{related} to each other in some way.

\end{document}

