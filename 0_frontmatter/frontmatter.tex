\documentclass[../main/main]{subfiles}


\begin{document}

\paragraph{Why study reinforcement learning?} Reinforcement learning is an exciting an active field of machine learning research. It has been used to solve a wide variety of problems, including robotics, game playing, and resource management. Reinforcement learning is also a powerful paradigm for studying animal and human behavior. In this book, we will focus on the problem of learning to make decisions in a sequential manner. This problem is a natural fit for many real-world problems, such as autonomous driving, robotics, and finance.


This book expects some knowledge of linear algebra and multivariable calculus. Students should be familiar with the following concepts:

\begin{itemize}
    \item \textbf{Linear Algebra:} Vectors, matrices, matrix multiplication, matrix inversion, eigenvalues and eigenvectors, and the Gram-Schmidt process.
    \item \textbf{Multivariable Calculus:} Partial derivatives, gradient, directional derivative, and the chain rule.
\end{itemize}


\section{Notation}

We will use the following notation throughout the book:

\todo{add notation}

\section{Challenges of reinforcement learning}

\paragraph{Exploration-exploitation tradeoff.} Should the agent try a new action or stick with the action that it knows is good?

\paragraph{Prediction.} The agent might want to predict the value of a state or state-action pair.

\paragraph{Policy computation (control).} In a complex environment, even if the dynamics are known, it can still be challenging to compute the best policy.

\end{document}
