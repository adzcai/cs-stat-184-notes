\documentclass[../main/main]{subfiles}

\setcounter{chapter}{1}

\begin{document}
    
\chapter{Markov Decision Processes}


\todo{Add section on ``What is RL?'' Agent taking actions that impact the environment. Introduce $\Delta$ notation. We'll also overload notation and use $s_t$ both to represent the actual state $s_t \in \S$ and to represent the \emph{event} that we observe state $s_t$ at time $t$. Most notation (e.g. uppercase letters for random variables) is borrowed from Sutton and Barto.}

How can we \emph{formalize} a reinforcement learning task in a way that is both \emph{sufficiently general} yet also tractable enough for \emph{fruitful analysis}?

In this chapter, we'll turn to \textbf{Markov decision processes} as a simple yet general formalism for solving decision problems.

\begin{definition}{Markov Decision Process}{mdp}
    
The key components of a Markov decision process are:

\begin{enumerate}
    \item The \textbf{state} (a.k.a. the \textbf{environment}) that the agent interacts with. We use $\S$ to denote the set of possible states, called the \textbf{state space}.
    \item The \textbf{agent} and the \textbf{actions} that it can take. We use $\A$ to denote the set of possible actions, called the \textbf{action space}.
    \item The \textbf{reward} signal. In this course we'll take it to be a deterministic function of a state-action pair, i.e. $r : \S \times \A \to \mathbb{R}$. In general, though, the reward function can also be stochastic, and it can also accept the \emph{resulting} state as an argument; that is, $r : \S \times \A \times \S \to \Delta(\R)$.
    \item The \textbf{state transitions} (a.k.a. \textbf{dynamics}) that describe what state we \textbf{transition to} after taking an action. We'll denote this by $P : \S \times \A \to \Delta(\S)$ (as opposed to $\P$ which denotes the underlying probability measure.)
    \item A \emph{discount factor} $\gamma \in [0, 1)$. We'll see later that this ensures that the \emph{return}, or total reward, is well-defined in infinite-horizon problems.
    \item Some \textbf{initial state distribution} $\rho \in \Delta(\S)$.
\end{enumerate}

Combined together, we call these a Markov decision process

\[
    M = (\S, \A, P, r, \gamma, \rho).
\]

\end{definition}


The reason we call it a \emph{Markov} decision process is that the transition function only depends on the ``current'' state and action. Formally, this implies that the state process satisfies the \textbf{Markov property}, that is,

\[
    \P(s_{t+1} \mid (s_\tau, a_\tau)_{\tau=0}^t) = P(s_{t+1} \mid s_t, a_t).
\]

\begin{example}{Examples of MDPs}{mdp_examples}
    \textbf{Board games and video games} are often MDPs. For example, in chess or Go, the state of the game only depends on the pieces on the board and not on the previous history. Several possible reward functions could be possible, e.g. $+1$ upon winning the game and $0$ otherwise, or to receive reward upon taking the opponent's pieces. The state transitions are based on the opponent's moves.

    \textbf{Robotic control} can be framed as an MDP task. In this setting, physics provides the state transitions. A possible action might be activating a motor to move forwards. The reward function could be designed based on the task; for example, one could reward the robot for arriving at a desired location.
\end{example}

We'll distinguish between \textbf{finite-horizon} MDPs, where the agent eventually enters a \textbf{terminal state}, and \textbf{infinite-horizon} MDPs, where the agent might keep going on and on.

We call the total reward the \emph{return}. For finite-horizon MDPs, we can just add up the rewards:

\[
    G_t := R_t + R_{t+1} + \cdots + R_T,
\]

where $T$ is the number of time steps and $R_t := r(S_t, A_t)$. However, for infinite-horizon problems (i.e. $T = \infty$), in order for this to be well-defined, we need to \emph{discount} future rewards:

\[
    G_t := R_t + \gamma R_{t+1} + \cdots + \gamma^{\tau-t} R_{\tau} + \cdots = \sum_{\tau = t}^\infty \gamma^{\tau-t} R_\tau.
\]

Can you see why this ensures that $G_t$ is finite?

Note that we recover the finite-horizon definition by letting $\gamma = 1$ and $T$ be finite.

Our key \emph{goal} in a reinforcement learning task is to \emph{maximize expected return}.

Why can't we just maximize the current reward at each timestep, i.e. use a greedy strategy? Well, in RL as in real life, often making greedy decisions (e.g. procrastinating) will leave you worse off than if you make some short-term sacrifices for long-term gains.

We call the ``video recording'' of states, actions, and rewards a \textbf{trajectory}

\[
    \xi_t = (s_\tau, a_\tau, r_\tau)_{\tau=0}^t
\]


\section{Policies and value functions}

A \textbf{policy} $\pi$ describes the agent's strategy: which actions it takes in a given situation.

Policies can either be \textbf{deterministic} (in the same situation, the agent will always take the same action) or \textbf{stochastic} (in the same situation, the agent will sample an action from a distribution).

What do I mean by ``situation''? In the most general setting, this could include all of the states, actions, and rewards in the trajectory so far.

However, due to the Markov assumption, the state transitions only depend on the current state. Thus a \textbf{stationary} policy $\pi : \S \to \Delta(\A)$ --- one that only depends on the current state --- can do just as well.

Fix a policy $\pi$. We'd like a concise way to refer to the expected return when \emph{starting in a given state} and acting according to $\pi$. We call this the \textbf{value function} of $\pi$ and denote it by

\begin{align*}
    V^\pi(s) &:= \E_\pi [G_0 \mid S_0 = s] % \\
    % &= \E_\pi \left[ \sum_{t=0}^\infty \gamma^t R_t \mid S_0 = s \right].
\end{align*}

We start at time $0$ without loss of generality; can you see why we could have chosen to start at any time?

Similarly, we can define the \textbf{action-value function} of $\pi$ (aka the \textbf{Q-function}) as the expected return when starting in a given state and taking a given action:

\begin{align*}
    Q^\pi(s, a) &:= \E_\pi [G_0 \mid S_0 = s, A_0 = a] % \\
    % &= \E_\pi [\sum_{t=0}^\infty \gamma^t R_t \mid S_0 = s, A_0 = a].
\end{align*}

\subsection{Bellman self-consistency equations}

Note that we can break down the return as

\[
    G_t = R_t + \gamma G_{t+1}:
\]

the reward from the \emph{current time-step} and that from \emph{future time-steps}. It turns out that this simple observation gives us a way to solve for the value function analytically!

Let's expand out the definition of the value function to see what I mean. Let's first consider the simple case where $\pi : \S \to \A$ is deterministic:

\begin{align*}
    V^\pi(s) &:= \E_\pi[G_0 \mid S_0 = s] \\
    &= r(s, \pi(s)) + \gamma \E_{s' \sim P(\cdot \mid s, \pi(s))} {\color{blue} \E_\pi [G_1 \mid S_1 = s']} \\
    &= r(s, \pi(s)) + \gamma \sum_{s' \in \S} P(s' \mid s, \pi(s)) {\color{blue} V^\pi(s')}.
\end{align*}

For stochastic policies, we simply average out over the relevant quantities:

\begin{align*}
    V^\pi(s) &:= \E_\pi [G_0 \mid S_0 = s] \\
    % &= \E_\pi \left[ \sum_{t=0}^\infty \gamma^t R_t \mid S_0 = s \right] \\
    &= \E_\pi \left[ R_0 + \gamma G_1 \mid S_0 = s \right] \\
    &= \E_{a \sim \pi(\cdot \mid s)} \left[ r(s, a) + \gamma \E_{s' \sim P(\cdot \mid s, a)}  {\color{blue} \E_\pi [G_1 \mid S_1 = s']} \right] \\
    &= \sum_a \pi(a \mid s) \left[r(s, a) + \gamma \sum_{s'} P(s' \mid s, a) {\color{blue} V^\pi(s')} \right].
\end{align*}




\section{Finite MDPs}

When the state and action space are finite, we can neatly express quantities as vectors and matrices:

\[
    r \in \R^{|\S| \times |\A|}, \quad P \in [0, 1]^{|\S| \times (|\S| \times |\A|)}, \quad \rho \in [0, 1]^{|\S|}, \quad \pi \in [0, 1]^{|\A| \times |\S|}, \quad V^\pi \in \R^{|\S|}, \quad Q^\pi \in \R^{|\S| \times |\A|}.
\]



\section{Exercises}

Show that without discounting, the reward


% For now, we'll assume that the world is known. This involves the state transitions and the reward.

% Unknown systems are similar to complex systems. In both, once we don't access the world everywhere, we need to actually \emph{learn} about the world around us.



\section{Optimality}


\begin{theorem}{Value Iteration}{val_iter}

Initialize:

\[
    V^0 \sim \|V^0\|_\infty \in [0, 1/1-\gamma]
\]

Iterate until convergence:

\[
    V^{t+1} \gets \mathcal{J}(V^t)
\]

\tcbsubtitle[colback=white]{Analysis}

This algorithm runs in $O(|\S|^3)$ time since we need to perform a matrix
inversion.

\end{theorem}





\begin{theorem}{Exact Policy Evaluation}{exact_pi_eval}

Represent the reward from each state-action pair as a vector

\[ R^\pi \in \R^{|\S|} \qquad R^\pi_s = r(s, \pi(s)) \]

Also represent the state transitions

\[ P^\pi \in \R^{|\S \times \S} \qquad P^\pi_{s, s'} = P(s' | s, \pi(s)) \]

That is, row $i$ of $P^\pi$ is a distribution over the \emph{next state}
given that the current state is $s_i$
and we choose an action using policy $\pi$.

Using this notation, we can express the Bellman consistency equation as

\begin{align*}
    \begin{pmatrix}
        \vdots \\ V^\pi(s) \\ \vdots
    \end{pmatrix}
    &=
    \begin{pmatrix}
        \vdots \\ r(s, \pi(s)) \\ \vdots
    \end{pmatrix}
    +
    \gamma
    \begin{pmatrix}
        & \vdots & \\
        \quad & P(s' \mid s, \pi(s)) & \quad \\
        & \vdots &
    \end{pmatrix}
    \begin{pmatrix}
        \vdots \\ V^\pi(s') \\ \vdots
    \end{pmatrix} \\
    V^\pi &= R^\pi + \gamma P^\pi V^\pi \\
    (I - \gamma P^\pi) V^\pi &= R^\pi \\
    V^\pi &= (I - \gamma P^\pi) R^\pi
\end{align*}

if $I - \gamma P^\pi$ is invertible, which we can prove is the case.


\end{theorem}



\begin{theorem}{Iterative Policy Evaluation}{iter_pi_eval}

How can we calculate the value function $V^\pi$ of a policy $\pi$?

Above, we saw an exact function that runs in $O(|\S|^2)$.
But say we really need a fast algorithm, and we're okay with having an
approximate answer. Can we do better? Yes!

Using the same notation as above,
let's initialize $V^0$ such that the elements are drawn uniformly
from $[0, 1/(1-\gamma)]$.

Then we can iterate the fixed-point equation we found above:

\[ V^{t+1} \gets R + \gamma P V^t \]


\end{theorem}

% One case in which we might want a fast, approximate algorithm like the one we
% just found is when we want to use it as part of another algorithm.

How can we use this fast approximate algorithm?

\begin{theorem}{Policy Iteration}{pi_iter}

Remember, for now we're only considering policies that are
\emph{stationary and deterministic}. There's $|\S|^{|\A}$ of these, so let's
start off by choosing one at random. Let's call this initial policy $\pi^0$,
using the superscript to indicate the time step.

Now for $t = 0, 1, \dots$, we perform the following:

\begin{enumerate}
    
\item \emph{Policy Evaluation}: First use the algorithm from earlier to
    calculate $V^{\pi^t}(s)$ for all states $s$. Then use this to calculate the
    state-action values:

    \[
        Q^{\pi^t}(s, a) = r(s, a) + \gamma \sum_{s'} P(s' \mid s, a) V^{\pi^t} (s')
    \]

\item \emph{Policy Improvement}: Update the policy so that, at each state,
    it chooses the action with the highest action-value:

    \[
        \pi^{t+1}(s) = \argmax_a Q^{\pi^t} (s, a)
    \]

    In other words, we're setting it to act greedily with respect to the new Q-function.

\end{enumerate}

What's the computational complexity of this?

% TODO

\end{theorem}




\section{Finite Horizon MDPs}

Suppose we're only able to act for $H$ timesteps.

% TODO come up with example

Now, instead of discounting, all we care about is the (average) total reward
that we get over this time.

\[ \E[ \sum_{t=0}^{H-1} r(s_t, a_t) ] \]

To be more precise, we'll consider policies that depend on the time.
We'll denote the policy at timestep $h$ as $\pi_h : \S \to \A$. In other
words, we're dropping the constraint that policies must be stationary.

This is also called an \emph{episodic model}.

% How to solve these problems? \emph{Dynamic programming}

% A bit more annoying notationally to keep track of time

Note that since our policy is nonstationary, we also need to adjust our value
function (and Q-function) to account for this.
Instead of considering the total infinite-horizon discounted reward like we did
earlier,
we'll instead consider the \emph{remaining} reward from a given timestep
onwards:

\begin{align*}
    V^\pi_h(s) &= \E \left[ \sum_\tau^{H-1} r(s_\tau, a_\tau) \mid s_h = s, a_\tau = \pi_h(s_h) \right] \\
    Q^\pi_h(s, a) &= \E \left[ \sum_\tau^{H-1} r(s_\tau, a_\tau) \mid (s_h, a_h) = (s, a) \right]
\end{align*}


We can also define our Bellman consistency equations, by splitting up the total
reward into the immediate reward (at this time step) and the future reward,
represented by our state value function from that next time step:

\[
    Q^\pi_h(s, a) = r(s, a) + \E_{s' \sim P(s, a)}[V^\pi_{h+1}(s')]
\]

\begin{theorem}{Computing the optimal policy}{pi_star_dp}

We can solve for the optimal policy using dynamic programming.

\begin{itemize}
\item \emph{Base case.} At the end of the episode (time step $H-1$),
    we can't take any more actions, so the $Q$-function is simply the reward
    that we obtain:

    \[
        Q^\star_{H-1}(s, a) = r(s, a)
    \]

    so the best thing to do is just act greedily
    and get as much reward as we can!

    \[
        \pi^\star_{H-1}(s) = \argmax_a Q^\star_{H-1}(s, a)
    \]

    Then $V^\star_{H-1}(s)$, the optimal value of state $s$ at the end of the
    trajectory, is simply whatever action gives the most reward.

    \[
        V^\star_{H-1} = \max_a Q^\star_{H-1}(s, a)
    \]

\item \emph{Recursion.} Then, we can work backwards in time, starting from the
    end, using our consistency equations!
\end{itemize}

Note that this is exactly just value iteration and policy iteration combined,
since our policy is nonstationary, so we can exactly specify its decisions at
each time step!

\tcbsubtitle[colback=white]{Analysis}


Total computation time $O(H |\S|^2 |\A|)$



\end{theorem}

\end{document}
